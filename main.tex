\documentclass{beamer}	
\mode<presentation>
 
\usepackage{pdfpages}
\usepackage{fancyvrb}
\usepackage{chemarr}

\usepackage{amsmath}		%% mathematics typesetting
\usepackage{amssymb}
 
\usepackage{epigraph}   %% nice setting of quotations

\usepackage{tabularx} %% allows to use row colours in tables

\usepackage{ulem}

\usepackage{booktabs}

\usepackage{siunitx} %% tpyeset SI units

\usepackage{CJKutf8} %% typeset Chinese characters

\usepackage{pdfpages}%% include pdfs

\usepackage{graphicx}
\usepackage{animate} %% show animated gifs

\DeclareMathAlphabet{\mathcalligra}{T1}{calligra}{m}{n}


% Color and Theme. Can be changed. However, this one's quite nice.
\usetheme{Madrid}
\definecolor{theme}{rgb}{0.84,0,0.21}
\usecolortheme[named=theme]{structure}

%%  Title information
\title[M11.12.2 Höhere motorische Zentren]{M11.12.2 Motorik 2: \\ Höhere motorische Zentren}
\author[melanie.stefan@medicalschool-berlin.de]{}
\institute[]{Prof. Melanie Stefan \\ melanie.stefan@medcialschool-berlin.de}
\date{SoSe 2022}
 

% Table of contents to pop up at the beginning of each section
\AtBeginSection[]
{
  \begin{frame}<beamer>
    \frametitle{Outline}
    \tableofcontents[currentsection,currentsubsection]
  \end{frame}
}
 
\beamertemplatenavigationsymbolsempty

\begin{document}


{ \usebackgroundtemplate{\includegraphics[width=1.2\paperwidth]{MSB_Titelseite.pdf}} 
\begin{frame}

 \maketitle 

$\,$\\[6cm] 


\end{frame} 
}


%% Hook: Aber was ist mit absichtlichen Bewegungen - nice picture with athlete

{ \usebackgroundtemplate{\includegraphics[width=1.2\paperwidth]{dancer.jpg}} 
\begin{frame}

$\;$\\[5cm]

\begin{flushright}
\textcolor{white}{Wie wird absichtliche \\ Bewegung gesteuert?}
\end{flushright}



\end{frame}
}

 
%% %% TLIA
{ \usebackgroundtemplate{\includegraphics[width=1.2\paperwidth]{dancer.jpg}} 
\begin{frame}{In dieser Vorlesung geht es um \dots}

$\;$\\[5cm]

\begin{flushright}
\textcolor{white}{\dots Kontrolle uns Ausführung \\ von willkürlichen Bewegungen}
\end{flushright}



\end{frame}
}



%% %% Learning Objectives
 
\begin{frame}

 \frametitle{Nach dieser Vorlesung sollten Sie folgendes können}



\begin{block}{Grundlagen:}
\begin{itemize}
\item
\end{itemize}

\end{block}



\end{frame}


 
\begin{frame}

 \frametitle{Nach dieser Vorlesung sollten Sie folgendes können}

\begin{block}{Klinik:}
\begin{itemize}
\item

\end{itemize}

\end{block}



\end{frame}










%% %% %% Main Body
 

 \section{Willkürbewegungen: Einführung}
 
 %% Definition: Picture Jogger
%  \begin{frame}{Willkürbewegungen}
 
 
% \begin{columns}[c]

% \begin{column}{5cm}

% Willkürbewegungen sind selbstinitiierte und selbstkontrollierte Bewegungen. \\

% Externe und interne Faktoren tragen zur Motivation einer Willkürbewegung bei. 

% \end{column}

% \begin{column}{5cm}



% \end{column}


% \end{columns}



 
     
%  \end{frame}
 
 
 %% Willkürbewegungen als gelernte Bewegungen
 
 %% Kette: Handlungsantrieb, Entschluss/Strategie, Programmierung, Umetzung
  \begin{frame}{Willkürbewegungen erfordern eine Kette von Vorgängen}
 
 \begin{itemize}
     \item \textbf{Handlungsantrieb} \\
     kortikale und subkortikale (z.B. limbische) Motivationsareale
     \item \textbf{Entschluss/Strategie} \\
     Basalganglien, kortikale Areale
          
     \item \textbf{Programmierung} \\
              Basalganglien, Kleinhirn, prämotorischer Kortex
     \item \textbf{Umsetzung} \\
     primärmotorischer Kortex, \(\alpha\)-Motorneurone, Skelettmuskel
 \end{itemize}

 \end{frame}
 
  \section{Basalganglien}
 
\begin{frame}{Baslalganglien kontrollieren Bewegungsabläufe}

Baslalganglien erleichtern erwünschte und unterdrücken unerwünschte Bewegungsabläufe. 


\begin{itemize}
    \item 
    Planung und Koordinierung von Willkürbewegungen
    \item
    Erstellung von Bewegungsprogrammen
    \item
    Situationsabhängige Auswahl von Bewegungsprogrammen
    \item
    Erlernen von Bewegungsabläufen 
    
\end{itemize}

    
\end{frame}
 
 
 %% Basalganglien: Struktur
\begin{frame}{Die Baslalganglien  sind ein
    neuronales Netzwerk von subkortikalen Kerngebieten}

\begin{center}
\includegraphics[width=\textwidth]{basalganglien.jpg}    
\end{center}

\end{frame}


%% Basalganglien: Aktivierungswege

\begin{frame}{Die Baslalganglien steuern Bewegung über zwei unterschiedliche Teilnetzwerke}

\begin{columns}[c]


\begin{column}{5cm}
 \begin{itemize}
        \item 
        Direkter Weg: Go (bewegungsfördernd)
        \item
        Indirekter Weg: NoGo (bewegungshemmend)
    \end{itemize}

\end{column}

% \begin{column}{5cmf}

% %% pictures: uncover basalganglien: all

% \end{column}


\end{columns}

\end{frame}



%% Striatum: Details

\begin{frame}{Striatum}

\begin{itemize}
\item
Das Striatum erhält Eingänge aus den Assoziationarealen und sensorischen Arealen des Kortex
    \item 
    Zu \(90\,\%\) mittelgroße Projecktionsneuronen (Medium Spiny Neurons, MSN)
    \item
    MSNs sind GABAerg
    \item
    2 verschiedene Peptid-Cotransmitter 
    \begin{itemize}
        \item 
        Substanz P: Go-Weg
        \item
        Enkaphalin: NoGo-Weg
        
    \end{itemize}
    \item
    MSN sind in Ruhe inaktiv
    \item
    Bei der Aktivierung spielen dopaminerge Eingänge aus der Substantia Nigra Pars Compacta (SNc) eine wichtige Rolle
    
\end{itemize}

\end{frame}




%% final version: comment from here
\begin{frame}{Regulierung des Striatums durch die Substantia Nigra Pars Compacta}

\begin{itemize}
\item
SNc wird aktiviert, wenn erlernte Bewegungsprogramme aufgerufen werden müssen oder bei unerwarteten Stimuli
    \item 
    Dopaminerge Signale von der SNc in das Striatum
    

\begin{itemize}
        \item D1 Rezeptoren: G\textsubscript{s} gekoppelt
        \item D2 Rezeptoren: G\textsubscript{i} gekoppelt
    \end{itemize}
    \textcolor{theme}{Was bedeutet das jeweils für cAMP in MSN?}    
\end{itemize}

\end{frame}
%% final version: comment to here




\begin{frame}{Regulierung des Striatums durch die Substantia Nigra Pars Compacta}

\begin{itemize}
\item
SNc wird aktiviert, wenn erlernte Bewegungsprogramme aufgerufen werden müssen oder bei unerwarteten Stimuli

    \item 
    Dopaminerge Signale von der SNc in das Striatum
    

\begin{itemize}
        \item D1 Rezeptoren: G\textsubscript{s} gekoppelt: \(\uparrow\) cAMP
        \item D2 Rezeptoren: G\textsubscript{i} gekoppelt: \(\downarrow\) cAMP
    \end{itemize}
    \pause
 \item
    Kopplung von Dopamin-Rezeptor und Ko-transmitter: \\ 
    D1 \(\rightarrow\) SP, D2 \(\rightarrow\) Enk  \\
    \pause
    Dopamin aus der SNc stimuliert den Go-Weg und hemmt den NoGo-Weg. 
    
\end{itemize}
\end{frame}


%% SNPC
% %% pictures: uncover basalganglien: all: full picture


%% Basalganglien: Funktionsschleifen
\begin{frame}{Funktionsschleifen der Basalganglien sorgen für Feedback}

\begin{block}{Skelettmotorische Schleife}
Prorgammierung und Initiierung von Körpermotorik \\
\textcolor{theme}{Globus pallidus internus} - Thalamus - Prämotorischer Kortex - \textcolor{theme}{Putamen}

\end{block}
\pause


\begin{block}{Okulomotorische Schleife}

Augenmotorik \\
\textcolor{theme}{Substantia nigra pars reticulata} - supplementär-motorische Augenfelder - frontales Augenfeld (Area 8) - \textcolor{theme}{Caudatum}


\end{block}
\end{frame}


\begin{frame}{Funktionsschleifen der Basalganglien sorgen für Feedback}

\begin{block}{Limbische Schleife}
Motivation, Handlungsantrieb \\

\textcolor{theme}{Globus pallidus} - Thalamus - zingulärer Kortex (+ Hippocampus?) - \textcolor{theme}{Globus pallidus}

\end{block}

\pause

\begin{block}{Kognitive (präfrontale) Schleife}

Kognitive Funktionen \\

\textcolor{theme}{Globus pallidus intnernus, Substantia nigra pars reticulata} - orbitofrontaler und präfrontaler Kortex - \textcolor{theme}{Striatum} 

\end{block}


    
\end{frame}


%% Morbus Parkinson
%% FINAL version_ comment from here
\begin{frame}{Morbus Parkinson}
\begin{columns}[c]

\begin{column}{5cm}

\begin{center}
    \includegraphics[width=\textwidth]{Lewy_Body_alphaSynuclein.jpg}
\end{center}

\end{column}

\begin{column}{5cm}

Aggregation von falsch gefaltetem \(\alpha\)-Synuclein zu Lewy Bodies in der SNc \\
Zellen der SNc sterben ab \\
\textcolor{theme}{Was sind die Auswirkungen?}

\end{column}


\end{columns}

    
\end{frame}
%% FINAL version_ comment to here



\begin{frame}{Morbus Parkinson}
\begin{columns}[c]

\begin{column}{5cm}

\begin{center}
    \includegraphics[width=\textwidth]{Lewy_Body_alphaSynuclein.jpg}
\end{center}

\end{column}

\begin{column}{5cm}

Aggregation von falsch gefaltetem \(\alpha\)-Synuclein zu Lewy Bodies in der SNc \\
Zellen der SNc sterben ab \\
Verlust des Dopamin-Signalling ins Striatum \\
Keine Aktivierung des Go-Weges, keine Hemmung des NoGo-Weges \\
Vermehrte Bewegungshemmung (Langsamkeit, Verlust von Koordination)
\end{column}


\end{columns}

\end{frame}



\begin{frame}{Chorea Huntington}


\begin{itemize}
    \item 
    Hyperkinetische Bewegungsstörung: Schnelle, unvorhersehbare Bewegungen von Kopf, Rumpf, Gesicht, und Extremitäten
    \item
    Häufigste genetisch bedingte neurologische Erkrankung, autosomal dominant vererbt
    \item
    Auftreten mit run 40 Jahren
    \item
    Zunächst: Degeneration der GABA/Enk MSN im Striatum: der NoGo-Weg geht verloren
    \item
    Später degenerieren zusätzlich auch GABA/SP MSN: komplexere Symptome (NoGo- und Go-Weg betroffen)
    
\end{itemize}


\end{frame}

 
 \section{Motorische Kortexareale}
 
 { \usebackgroundtemplate{\includegraphics[width=1\paperwidth]{motorkortex.png}} 
\begin{frame}



\end{frame} 
}


\begin{frame}{Primärmotorischer Kortex (M1, Area 4)}

\begin{itemize}
    \item 
    Schicht IV: thalamischer Eingang, agranulär
    \item
    Pyramidalaxone aus Schicht III und V bilden \(40\,\%\) des Tractus Corticospinalis = Pyramidenbahnen
    \item
    Betzsche Riesenzellen aus Schicht V (Pyramidalzellen) senden Axone   zu \(\alpha\)-Motoneuronen im Rückenmark \\
    vor allem Kontrolle der Finger (Feinmotorik)
    \item
    Assoziationsfasern zur ipsilateralen und kontralateralen Hirnrinde
    \item
    Subkortikale Projektionen: 
    \begin{itemize}
        \item 
        Zum Stammhirn: Schleife mit Basalganglien (kortiko-thalamo-kortikale Schleife)
        \item
        Zu den pontinen Nuclei: Schleife mit Cerebellum
    \end{itemize}
\end{itemize}

\end{frame}


\begin{frame}{Primärmotorischer Kortex (M1, Area 4)}

\begin{columns}[c]

\begin{column}{5cm}
Somatotope Organisation (motorischer Homunculus) \\

Kontralaterale Repräsentation (Ausnahmen: Minische Muskulatur Stirnregion, Zungen-, Kau- und Schluckmuskulatur sind bilateral repräsentiert) \\

\pause

Organisiert in Säulen: Neuronen innherhalb einer Säule kommunizieren miteinander: Mehrere Säulen bilden eine funktionelle Einheit  \\

In M1 werden Bewegungsprogramme durchgeführt und an spinale Neurone weitergeleitet \\

\pause

Entladungsfrequenzen von M1-Neuronen korrelieren mit Kontraktionsstärke Muskel, Richung und Dynamik der Bewegung.  \\

Rückmeldung durch popriozeptive und kutane Afferenzen für Anpassung der Kraft (transkortikale Reflexe) \\

 Läsionen des M1 (z.B. durch Schlaganfall) führen zu Lähmungen (Paresen) 


\end{column}

\begin{column}{5cm}
\begin{center}
    \includegraphics[width=\textwidth]{Motor_homunculus.png}
\end{center}

\end{column}


\end{columns}
\end{frame}



\begin{frame}{Prämotorischer Kortex (Area 6)}

\begin{itemize}
\item
Somatotrop organisiert
\item
Erstellt Bewegungsprogramme zur Weiterleitung an M1
\item
Afferenzen:
\begin{itemize}
    \item 
    Motorischer Thalamus (verarbeitet Informationen von Basalganglien und Kleinhirn)
    \item
    Sensorischer Thalamus (verarbeitet Informationen von Hirnstamm und Rückenmark)
    \item
    Zingulärer Kortex
\end{itemize}
\item
Efferenzen: 
\begin{itemize}
    \item 
    Primär-motorischer Kortex
    \item
    Pontine Nuclei
    \item
    Motorischer Hirnstamm (v.a. Nucleus ruber, Formatio reticularis)
\end{itemize}
\item
Einteilung in SMA und PMA
\end{itemize}

\end{frame}

\begin{frame}{Supplementär-motorische Area (SMA)}


\begin{itemize}
    \item 
    Starterfunktion ür intrinsisch initiierte Bewegungen
    \item
    Auswahl und Durchführung angemessener Willkürbewegungen
    \item
    Aktiv beim Vorstellen von Bewegungen
    \item
    Bimanuelle Koordination
\end{itemize}
    
\end{frame}

\begin{frame}{Prämotorische Area (PMA)}

\begin{columns}[c]

\begin{column}{5cm}

\begin{itemize}
    \item 
    Extrinsich initiierte Bewegungen, visuelle oder kutane Stimulierung
    \item
    Assoziiert physikalsiche Eigenschaften eines Objects mit spezifischen motorischen Handlungen 
    \item
    Antizipatorische Kraftkontrolle ("Wie schwer ist das Object, das ich heben möchte")
    \item
    Viso-motorische Steuerung des Greifens: Afferenzen vom posterior-parietalen Kortex, Efferenzen zum primär-motorischen Kortex
    \item
    Spiegelneuronen (imitatives Lernen)
\end{itemize}
\end{column}

\begin{column}{5cm}

\begin{center}
    \includegraphics[width=\textwidth]{water.jpg}
\end{center}


\end{column}
\end{columns}

\end{frame}




\begin{frame}{Frontales Augenfeld  (Area 8)}

\begin{itemize}
    \item 
    Willkürliche Augenbewegung
    \item
    Projiziert zum Zwischen- und Mittelhirn $\rightarrow$ motorische Augenmuskelkerne III, IV, VI
    \item
    Bei Läsionen ist willkürliche Augenbewegung nicht mehr möglich
\end{itemize}

\end{frame}


\begin{frame}{Kleiner Selbsttest}

\textcolor{theme}{Welcher Teil des Motorkortex macht hier was?}

\begin{center}
    \includegraphics[width=\textwidth]{tennis.jpg}
\end{center}


\end{frame}


 \section{Der motorische Hirnstamm}

\begin{frame}{Motorischer Hirnstamm}
    
    \begin{itemize}
        \item 
        
    Essentiell für aufrechte Körperhaltung und Stützmotorik
    \item
    Sensorische und motorische Funktionen Gesicht und Kopf (analog zum Rückenmark für den Rest des Körpers)
    \item
    Kontrolle Körperfunktionen: Atmung, kardiovaskuläre  Funktionen
    \item
    Relais-Station für deszendierende Bahnen
    \end{itemize}

    
\end{frame}

\begin{frame}{Vom Hirnstamm deszendierende Bahnen sich für die Stützmotorik wichtig}

\begin{block}{Obere Extremitäten}
Erregen Flexoren, hemmen Strecker
    \begin{itemize}
        \item 
        Tractus reticulospinalis, aus der  Formatio reticularis (Medulla)
        \item
        Tratus rubrospinalis, aus dem  Nucelus ruber
    \end{itemize}
\end{block}


\begin{block}{Untere Extremitäten}
Hemmen Flexoren, erregen Strecker
\begin{itemize}
        \item 
        Tractus vestibulospinalis, aus demNucleus vestibularis
        \item
        Tractus reticularis medialis, aus der Formatio reticularis (Pons)
        \end{itemize}


\end{block}


Supraspinale Zentren hemmen die Aktivität dieser Bahnen
    
\end{frame}


 
\section{Deszendierende Bahnen}

 \section{Eingänge in motorische Strukturen }

 \section{Zusammenspiel motorischer Strukturen bei Willkürbewegungen}

 


%% %% %% %% Review

\begin{frame}

 \frametitle{Jetzt* sollten Sie folgendes können}



\end{frame}




%% %% %% %% Feedbackhinweisblock

\begin{frame}
\frametitle{Danke für Ihr Feedback!}

\begin{columns}[c]

\begin{column}{6cm}
\begin{center}
 \includegraphics[width=\textwidth]{smilie_balloons.jpg}
\end{center}

\end{column}

\begin{column}{4cm}


\begin{center}
\includegraphics[width=\textwidth]{feedback_QR.png}
\end{center}
\end{column}


\end{columns}
\end{frame}


%% %% %% Bildnachweis
\begin{frame}
\frametitle{Bildnachweis}
\begin{tiny}

Teile dieser Vorlesung wurden übernommen von einer Vorlesung von Prof. Maike Glitsch, Medical School Hamburg, der wir an dieser Stelle herzlich danken. Wo nicht anders gekennzeichnet, stammen Abbildungen aus dieser Vorlesung.  


 
\begin{itemize}


\item
Basalganglien. Von User:Bascon - Eigenes Werk, Gemeinfrei, \url{https://commons.wikimedia.org/w/index.php?curid=961906}


\item
Joggende Person. Photo by \href{https://unsplash.com/@sagefriedman?utm_source=unsplash&utm_medium=referral&utm_content=creditCopyText}{Sage Friedman} on \href{https://unsplash.com/s/photos/runner?utm_source=unsplash&utm_medium=referral&utm_content=creditCopyText}{Unsplash}
  
\item
Lewy Body. By Marvin 101 - Own work, CC BY-SA 3.0, \url{https://commons.wikimedia.org/w/index.php?curid=7533521}

%% all lectures
\item
Luftballons mit frohen und traurigen Smilies. Photo by \href{https://unsplash.com/@artbyhybrid?utm_source=unsplash&utm_medium=referral&utm_content=creditCopyText}{Hybrid} on \href{https://unsplash.com/s/photos/feedback?utm_source=unsplash&utm_medium=referral&utm_content=creditCopyText}{Unsplash}
%%%%%%%%%%%


\item 
Motorischer Homunculus. By mailto:ralf@ark.in-berlin.de - File:Homunculus-ja.pngAfter Penfield and Rasmussen (1950), The Cerebral Cortex of Man.Modified from (an earlier version of) File:Homunculus-de.png by Was a bee., CC BY-SA 4.0, \url{https://commons.wikimedia.org/w/index.php?curid=49377875}

\item
Person schenkt sich Wasser ein. Photo by \href{https://unsplash.com/@sanengineer?utm_source=unsplash&utm_medium=referral&utm_content=creditCopyText}{Ikhsan Sugiarto} on \href{https://unsplash.com/s/photos/water-glass?utm_source=unsplash&utm_medium=referral&utm_content=creditCopyText}{https://unsplash.com/@sanengineer?utm_source=unsplash&utm_medium=referral&utm_content=creditCopyText}
  


\item
Tanzende Person. Photo by \href{https://unsplash.com/es/@aoddeh?utm_source=unsplash&utm_medium=referral&utm_content=creditCopyText}{Ahmad Odeh} on \href{https://unsplash.com/s/photos/dance?utm_source=unsplash&utm_medium=referral&utm_content=creditCopyText}{Unsplash}
  

\item
Teilnetzwerke der Basalganglien. Von Scarecr0w 4 - Eigenes Werk, Gemeinfrei, \url{https://commons.wikimedia.org/w/index.php?curid=6974431}

\item
Tennisspieler. Photo by \href{https://unsplash.com/@chinorocha?utm_source=unsplash&utm_medium=referral&utm_content=creditCopyText}{Chino Rocha} on \href{https://unsplash.com/s/photos/tennis?utm_source=unsplash&utm_medium=referral&utm_content=creditCopyText}{Unsplash}
  


\end{itemize}
\end{tiny}
\end{frame}






\end{document}

%%% Frequently used snippets

%% \begin{columns}[c]

%% \begin{column}{5cm}
%% \end{column}

%% \begin{column}{5cm}
%% \end{column}


%% \end{columns}




