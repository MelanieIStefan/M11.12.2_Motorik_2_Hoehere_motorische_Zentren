\documentclass{beamer}	
\mode<presentation>
 
\usepackage{pdfpages}
\usepackage{fancyvrb}
\usepackage{chemarr}

\usepackage{amsmath}		%% mathematics typesetting
\usepackage{amssymb}
 
\usepackage{epigraph}   %% nice setting of quotations

\usepackage{tabularx} %% allows to use row colours in tables

\usepackage{ulem}

\usepackage{booktabs}

\usepackage{siunitx} %% tpyeset SI units

\usepackage{CJKutf8} %% typeset Chinese characters

\usepackage{pdfpages}%% include pdfs

\usepackage{graphicx}
\usepackage{animate} %% show animated gifs

\DeclareMathAlphabet{\mathcalligra}{T1}{calligra}{m}{n}


% Color and Theme. Can be changed. However, this one's quite nice.
\usetheme{Madrid}
\definecolor{theme}{rgb}{0.84,0,0.21}
\usecolortheme[named=theme]{structure}

%%  Title information
\title[M11.11.5 Lernen, Gedächtnis]{M11.11.5 Zentrales Nervensystem 3: \\ Lernen und Gedächtnis}
\author[melanie.stefan@medicalschool-berlin.de]{}
\institute[]{Prof. Melanie Stefan \\ melanie.stefan@medcialschool-berlin.de}
\date{SoSe 2022}
 

% Table of contents to pop up at the beginning of each section
\AtBeginSection[]
{
  \begin{frame}<beamer>
    \frametitle{Outline}
    \tableofcontents[currentsection,currentsubsection]
  \end{frame}
}
 
\beamertemplatenavigationsymbolsempty

\begin{document}


{ \usebackgroundtemplate{\includegraphics[width=1\paperwidth]{MSB_Titelseite.pdf}} 
\begin{frame}

 \maketitle 

$\,$\\[6cm] 


\end{frame} 
}


%% Hook: Aber was ist mit absichtlichen Bewegungen - nice picture with athlete

% { \usebackgroundtemplate{\includegraphics[width=1.2\paperwidth]{miklos-kornyei-pWzCAm4jgzA-unsplash.jpg}} 
% \begin{frame}

% \end{frame}
% }

 
%% %% TLIA



%% %% Learning Objectives
 
\begin{frame}

 \frametitle{Nach dieser Vorlesung sollten Sie folgendes können}



\begin{block}{Grundlagen:}
\begin{itemize}
\item
\end{itemize}

\end{block}



\end{frame}


 
\begin{frame}

 \frametitle{Nach dieser Vorlesung sollten Sie folgendes können}

\begin{block}{Klinik:}
\begin{itemize}
\item

\end{itemize}

\end{block}



\end{frame}










%% %% %% Main Body
 

 \section{Willkürbewegungen: Einführung}
 
 %% Definition
 
 %% Willkürbewegungen als gelernte Bewegungen
 
 %% Kette: Handlungsantrieb, Entschluss/Strategie, Programmierung, Umetzung
  \begin{frame}{Willkürbewegungen erfordern eine Kette von Vorgängen}
 
 \begin{itemize}
     \item \textbf{Handlungsantrieb} \\
     kortikale und subkortikale (z.B. limbische) Motivationsareale
     \item \textbf{Entschluss/Strategie} \\
     Basalganglien, kortikale Areale
          
     \item \textbf{Programmierung} \\
              Basalganglien, Kleinhirn, prämotorischer Kortex
     \item \textbf{Umsetzung} \\
     primärmotorischer Kortex, \(\alpha\)-Motorneurone, Skelettmuskel
 \end{itemize}

 \end{frame}
 
  \section{Basalganglien}
 
\begin{frame}{Baslalganglien kontrollieren Bewegungsabläufe}

Baslalganglien erleichtern erwünschte und unterdrücken unerwünschte Bewegungsabläufe. 


\begin{itemize}
    \item 
    Planung und Koordinierung von Willkürbewegungen
    \item
    Erstellung von Bewegungsprogrammen
    \item
    Situationsabhängige Auswahl von Bewegungsprogrammen
    \item
    Erlernen von Bewegungsabläufen 
    
\end{itemize}

    
\end{frame}
 
 
 %% Basalganglien: Struktur
\begin{frame}{Die Baslalganglien  sind ein
    neuronales Netzwerk von subkortikalen Kerngebieten}

\begin{center}
\includegraphics[width=\textwidth]{basalganglien.jpg}    
\end{center}

\end{frame}


%% Basalganglien: Aktivierungswege

\begin{frame}{Die Baslalganglien steuern Bewegung über zwei unterschiedliche Teilnetzwerke}

\begin{columns}[c]


\begin{column}{5cm}
 \begin{itemize}
        \item 
        Direkter Weg: Go (bewegungsfördernd)
        \item
        Indirekter Weg: No Go (bewegungshemmend)
    \end{itemize}

\end{column}

% \begin{column}{5cmf}

% %% pictures: uncover basalganglien: all

% \end{column}


\end{columns}

\end{frame}



%% Striatum: Details

\begin{frame}{Striatum}

\begin{itemize}
\item
Das Striatum erhält Eingänge aus den Assoziationarealen und sensorischen Arealen des Kortex
    \item 
    Zu \(90\,\%\) mittelgroße Projecktionsneuronen (Medium Spiny Neurons, MSN)
    \item
    MSNs sind GABAerg
    \item
    2 verschiedene Peptid-Cotransmitter 
    \begin{itemize}
        \item 
        Substanz P: Go-Weg
        \item
        Enkaphalin: No Go-Weg
        
    \end{itemize}
    \item
    MSN sind in Ruhe inaktiv
    \item
    Bei der Aktivierung spielen dopaminerge Eingänge aus der Substantia Nigra Pars Compacta (SNc) eine wichtige Rolle
    
\end{itemize}

\end{frame}




%% final version: comment from here
\begin{frame}{Regulierung des Striatums durch die Substantia Nigra Pars Compacta}

\begin{itemize}
\item
SNc wird aktiviert, wenn erlernte Bewegungsprogramme aufgerufen werden müssen oder bei unerwarteten Stimuli
    \item 
    Dopaminerge Signale von der SNc in das Striatum
    
\end{itemize}
\begin{itemize}
        \item D1 Rezeptoren: G\textsubscript{s} gekoppelt
        \item D2 Rezeptoren: G\textsubscript{i} gekoppelt
    \end{itemize}
    \textcolor{theme}{Was bedeutet das jeweils für cAMP in MSN?}    
\end{frame}
%% final version: comment to here




\begin{frame}{Regulierung des Striatums durch die Substantia Nigra Pars Compacta}

\begin{itemize}
\item
SNc wird aktiviert, wenn erlernte Bewegungsprogramme aufgerufen werden müssen oder bei unerwarteten Stimuli

    \item 
    Dopaminerge Signale von der SNc in das Striatum
    
\end{itemize}
\begin{itemize}
        \item D1 Rezeptoren: G\textsubscript{s} gekoppelt: \(\uparrow\) cAMP
        \item D2 Rezeptoren: G\textsubscript{i} gekoppelt: \(\downarrow\) cAMP
    \end{itemize}
    \pause
 \item
    Kopplung von Dopamin-Rezeptor und Ko-transmitter: D1 \(\rightrarrow\) SP, D2 \(\rightrarrow\) Enk  \\
    \pause
    \(\rightarrow\) Dopamin aus der SNc stimuliert den Go-Weg und hemmt den NoGo-Weg. 
    

\end{frame}


%% SNPC
% %% pictures: uncover basalganglien: all: full picture


%% Basalganglien: Funktionsschleifen
% \begin{frame}{Funktionsschleifen der Basalganglien}

% \begin{block}{Skelettmotorische Schleife}
% Prorgammierung und Initiierung von Körpermotorik
% \end{block}

% \begin{block}{Okulomotorische Schleife}

% \end{block}


% \begin{block}{Limbische Schleife}

% \end{block}


% \begin{block}{Kognitive (präfrontale) Schleife}

% \end{block}


    
% \end{frame}




 
 \section{Motorische Kortexareale}
 
 
 \section{Der motorische Hirnstamm}
 
\section{Deszendierende Bahnen}
 
 \section{Eingänge in motorische Strukturen }

 \section{Zusammenspiel motorischer Strukturen bei Willkürbewegungen}

 


%% %% %% %% Review

\begin{frame}

 \frametitle{Jetzt* sollten Sie folgendes können}



\end{frame}




%% %% %% %% Feedbackhinweisblock

\begin{frame}
\frametitle{Danke für Ihr Feedback!}

\begin{columns}[c]

\begin{column}{6cm}
\begin{center}
 \includegraphics[width=\textwidth]{smilie_balloons.jpg}
\end{center}

\end{column}

\begin{column}{4cm}


\begin{center}
% \includegraphics[width=\textwidth]{feedback_QR.png}
\end{center}
\end{column}


\end{columns}
\end{frame}


%% %% %% Bildnachweis
\begin{frame}
\frametitle{Bildnachweis}
\begin{tiny}

% Teile dieser Vorlesung wurden übernommen von einer Vorlesung von Prof. Emanuel Busch,  Health and Medical University Potsdam, dem wir an dieser Stelle herzlich danken. Wo nicht anders gekennzeichnet, stammen Abbildungen aus dieser Vorlesung.  


 
\begin{itemize}


\item
Basalganglien. Von User:Bascon - Eigenes Werk, Gemeinfrei, \url{https://commons.wikimedia.org/w/index.php?curid=961906}


%% all lectures
\item
Luftballons mit frohen und traurigen Smilies. Photo by \href{https://unsplash.com/@artbyhybrid?utm_source=unsplash&utm_medium=referral&utm_content=creditCopyText}{Hybrid} on \href{https://unsplash.com/s/photos/feedback?utm_source=unsplash&utm_medium=referral&utm_content=creditCopyText}{Unsplash}
%%%%%%%%%%%

\item
Teilnetzwerke der Basalganglien. Von Scarecr0w 4 - Eigenes Werk, Gemeinfrei, \url{https://commons.wikimedia.org/w/index.php?curid=6974431}




\end{itemize}
\end{tiny}
\end{frame}






\end{document}

%%% Frequently used snippets

%% \begin{columns}[c]

%% \begin{column}{5cm}
%% \end{column}

%% \begin{column}{5cm}
%% \end{column}


%% \end{columns}




